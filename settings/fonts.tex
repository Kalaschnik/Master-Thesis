\usepackage{pifont}					% Symbole nutzen z. B. Aldusblatt mit \ding{167}
\usepackage{fontawesome}            % Font Awesome Icons nutzen

% http://ftp.fau.de/ctan/fonts/ccicons/ccicons.pdf
% \ccShareAlike
\usepackage[scale=0.9]{ccicons}


\usepackage{eurosym}				% sauberes Eurosymbol, siehe: http://www.theiling.de/eurosym.html


\usepackage[oldstyle]{libertine} % set typewriter a bit smaller to get a nicer typewriter size, oldstyle set OsF for all numbers
% use the two commands for Th and Qu ligatures. Leave Libtertine under all circumstances at T1
% be aware you will not be able to find the word Query if you use these ligatures... etc.
\newcommand{\Th}{\begingroup\fontencoding{OT1}\selectfont Th\endgroup}
\newcommand{\Qu}{\begingroup\fontencoding{OT1}\selectfont Qu\endgroup}

\usepackage{CJKutf8} % Chinese-Japanese-Korean 
% use with: \begin{CJK}{UTF8}{min}未練なく散も桜はさくら哉\end{CJK}

\usepackage{libertinust1math}

\usepackage[scaled=.92]{helvet}	% Helvetica-Font for headings

\usepackage[scaled=0.83]{beramono}

% change single levels to appear differently:
\setkomafont{subparagraph}{\normalfont\textit}
% \setkomafont{subparagraph}{\normalfont\textbf}
% \RedeclareSectionCommand[indent=0pt]{subparagraph}
% \RedeclareSectionCommands[
%     beforeskip=-3.25ex plus -1ex minus -0.2ex,
%     runin=false,
%     afterskip=-\parskip
% ]{paragraph,subparagraph}

% this makes ligatures searchable in the pdf file
\input{glyphtounicode}
\pdfglyphtounicode{f_f}{FB00}
\pdfglyphtounicode{f_f_i}{FB03}
\pdfglyphtounicode{f_f_l}{FB04}
\pdfglyphtounicode{f_i}{FB01}
\pdfgentounicode=1