\usepackage{scrhack}

\usepackage[hyphens]{url}

\usepackage{censor} % allows to \censor{certain} text

\usepackage[usenames,dvipsnames,svgnames,table]{xcolor} % https://en.wikibooks.org/wiki/LaTeX/Colors

\usepackage[most]{tcolorbox}
% create color boxes around text 
% https://tex.stackexchange.com/questions/327323/how-to-change-options-in-the-fcolorbox-environment

\usepackage[absolute,overlay]{textpos} % allows to free position text, as such:
% \begin{textblock*}{5cm}(10cm,6cm) % {block width} (coords) 
%   Your text here
% \end{textblock*}

\usepackage{float}		% enables H for tab/figs: \begin{table}[H]
% if you don't have a blank line after \end{xxx}[H] there will be no indentation:
\makeatletter

\renewcommand\float@endH{\@endfloatbox\vskip\intextsep
  \if@flstyle\setbox\@currbox\float@makebox\columnwidth\fi
  \box\@currbox\vskip\intextsep\relax\@doendpe}

\makeatother


\usepackage{tikz} % advanced grpahics such as figure numbering etc.
\newcommand*\circled[1]{\tikz[baseline=(char.base)]{
		\node[shape=circle,draw,inner sep=1.75pt] (char) {#1};}} % make circled numbers with \circled{1} etc
\usetikzlibrary{shadows.blur}



\usepackage{relsize} % wird für \textscale verwendet
\newcommand*{\tracknshrink}[1]{\textscale{0.9}{\textls*[25]{#1}}} % \tracknshrink{UNO}: verkleinert und sperrt Versalabkürzungen
\newcommand*{\tns}[1]{\textscale{0.9}{\textls*[25]{#1}}} % \tracknshrink{UNO}: verkleinert und sperrt Versalabkürzungen


\usepackage{enumitem}	% allows you to change the left margin on e.g. \begin{enumerate}[leftmargin=2cm]
% override single enviroments with: \begin{itemize}[label={-}]
\renewcommand{\labelitemii}{\raisebox{0.37mm}{\scalebox{.5}{$\blacksquare$}}}
\renewcommand{\labelitemiii}{\raisebox{0.37mm}{\scalebox{.5}{$\blacksquare$}}}

\usepackage{caption}				% fixt den scrollbug bei ref-Links für captions (fig & tables)

\usepackage{graphicx}			% benötigt für Grafiken
\graphicspath{{./img/}}			% global graphics path

\usepackage{booktabs}			% See myriad of alternatives: http://tex.stackexchange.com/questions/12672/which-tabular-packages-do-which-tasks-and-which-packages-conflict
% usage: http://www.namsu.de/Extra/pakete/Booktabs.html
\usepackage{multirow} % use multirow for tables
\newcommand*\rot{\rotatebox{90}} % use \rot to rotate text 90° for tables e.g.

\usepackage{csquotes}			% glossaries needs this


\usepackage{xargs}                      % Use more than one optional parameter in a new commands
\usepackage[colorinlistoftodos,prependcaption,bordercolor=yellow!75,backgroundcolor=yellow!40,linecolor=yellow!75,textsize=scriptsize,textwidth=1.95cm]{todonotes}
\newcommandx{\unsure}[2][1=]{\todo[linecolor=Turquoise!30,backgroundcolor=Turquoise!20,bordercolor=Turquoise!30,#1]{#2}}
\newcommandx{\wording}[2][1=]{\todo[linecolor=blue!30,backgroundcolor=blue!20,bordercolor=blue!30,#1]{#2}}
\newcommandx{\info}[2][1=]{\todo[linecolor=Goldenrod!30,backgroundcolor=Goldenrod!20,bordercolor=Goldenrod!30,#1]{#2}}
\newcommandx{\citethis}[2][1=]{\todo[linecolor=BrickRed!30,backgroundcolor=BrickRed!20,bordercolor=BrickRed!30,#1]{#2}}
\newcounter{userstoryc}
\newcommand*\lect{%
   \stepcounter{lecture}%
   \marginpar{\fbox{Lecture \thelecture}}}
\newcommand{\userstory}[1]{%
   \stepcounter{userstoryc}%
   \paragraph{Story \#\theuserstoryc}%
   \emph{Priority:} #1\\}
   
   
   
\usepackage{lettrine} % fro drop caps

\usepackage{tabto}
% The quick \tabto{3.5cm}brown \tabto{5.5cm}fox


% define below/above reference using \here{keyword} \where{keyword}
\makeatletter
\newcount\here@undef
\newcommand{\here}[1]{%
  \@ifundefined{here@#1@undef}{}{\advance\here@undef by -1}%
  \@namedef{here@#1}{}}
\newcommand{\where}[1]{%
  \@ifundefined{here@#1}{%
    below%
    \@ifundefined{here@#1@undef}{%
      \@namedef{here@#1@undef}{}%
      \advance\here@undef by 1
    }{}%
  }{%
    above%
  }%
}
\AtEndDocument{%
  \ifnum\here@undef>0
    \GenericWarning{}{There were undefined above/below labels}%
  \fi}
\makeatother




\usepackage[edges]{forest} % for any kind of trees or directory listings
\definecolor{folderbg}{RGB}{0,0,0}
\definecolor{folderborder}{RGB}{128,128,128}
\newlength\Size
\setlength\Size{4pt}
\tikzset{%
  folder/.pic={%
    \filldraw [draw=folderborder, top color=folderbg!50, bottom color=folderbg] (-1.05*\Size,0.2\Size+5pt) rectangle ++(.75*\Size,-0.2\Size-5pt);
    \filldraw [draw=folderborder, top color=folderbg!50, bottom color=folderbg] (-1.15*\Size,-\Size) rectangle (1.15*\Size,\Size);
  },
  file/.pic={%
    \filldraw [draw=folderborder, top color=folderbg!5, bottom color=folderbg!10] (-\Size,.4*\Size+5pt) coordinate (a) |- (\Size,-1.2*\Size) coordinate (b) -- ++(0,1.6*\Size) coordinate (c) -- ++(-5pt,5pt) coordinate (d) -- cycle (d) |- (c) ;
  },
}
\forestset{%
  declare autowrapped toks={pic me}{},
  pic dir tree/.style={%
    for tree={%
      folder,
      font=\sffamily,
      grow'=0,
    },
    before typesetting nodes={%
      for tree={%
        edge label+/.option={pic me},
      },
    },
  },
  pic me set/.code n args=2{%
    \forestset{%
      #1/.style={%
        inner xsep=2\Size,
        pic me={pic {#2}},
      }
    }
  },
  pic me set={directory}{folder},
  pic me set={file}{file},
}






% define a counter
% fr = functional requirements
\newcounter{fr}
% nfr = non-functional requirements
\newcounter{nfr}

% setspace for toc spacing
% https://latex.org/forum/viewtopic.php?t=4628
\usepackage{setspace}