\chapter*{Abstract}


\noindent In the field of Semantic Web, much effort is invested in developing possible solutions for exploring and managing graph data in a visual context. eccenca’s Corporate Memory, an enterprise application suite, enables users to work with semantic models and allows intuitive data exploration. The current work uses eccenca’s software infrastructure to develop a novel approach to visualize \tns{RDF} resources by mapping graph data in a Kanban board. Based on an \tns{RDF} configuration graph, users can select the resources represented as the cards of the Kanban board, as well as the property used to represent the columns of the board. In addition to this visualization, the developed prototype allows to modify a resource by the prior selected column property when moving a card between columns on the board. Dropping a card to a novel column triggers the resource to update its property to the value of this column. Visualizing and manipulating knowledge data by relocating cards on a Kanban board represents a innovative approach in the field of semantic data exploration.

\vspace{0.5cm}
{\textit{Keywords}\small\quad Kanban Board, React, JavaScript, Semantic Web, \tns{SPARQL}, \tns{RDF(S)}, \tns{OWL}, \tns{SHACL}}